\documentclass{mproj}
\usepackage{graphicx}
\usepackage{subcaption}
\usepackage{subfig}
\usepackage{graphicx}
\usepackage{url}
\usepackage{fancyvrb}
\usepackage[final]{pdfpages}
\usepackage{subfloat}
\usepackage{floatrow}
\usepackage{wrapfig}
\usepackage{picinpar}
\usepackage{fancyref}

\graphicspath{{images/}}

% alternative font if you prefer
\renewcommand{\familydefault}{\sfdefault}
\usepackage{helvet}
\usepackage[T1]{fontenc}
\usepackage{textcomp}

% for alternative page numbering use the following package
% and see documentation for commands
%\usepackage{fancyheadings}


% other potentially useful packages
%\uspackage{amssymb,amsmath}
%\usepackage{url}
%\usepackage{fancyvrb}
%\usepackage[final]{pdfpages}

\begin{document}

%%%%%%%%%%%%%%%%%%%%%%%%%%%%%%%%%%%%%%%%%%%%%%%%%%%%%%%%%%%%%%%%%%%
\title{Gold Digger: a searching behaviour game}
\author{Gabriele Giordano Maria Rossi}
\date{Date of submission placed here}
\maketitle
%%%%%%%%%%%%%%%%%%%%%%%%%%%%%%%%%%%%%%%%%%%%%%%%%%%%%%%%%%%%%%%%%%%

%%%%%%%%%%%%%%%%%%%%%%%%%%%%%%%%%%%%%%%%%%%%%%%%%%%%%%%%%%%%%%%%%%%
\begin{abstract}
abstract goes here
\end{abstract}
%%%%%%%%%%%%%%%%%%%%%%%%%%%%%%%%%%%%%%%%%%%%%%%%%%%%%%%%%%%%%%%%%%%

%%%%%%%%%%%%%%%%%%%%%%%%%%%%%%%%%%%%%%%%%%%%%%%%%%%%%%%%%%%%%%%%%%%
\educationalconsent

%%%%%%%%%%%%%%%%%%%%%%%%%%%%%%%%%%%%%%%%%%%%%%%%%%%%%%%%%%%%%%%%%%%

\newpage
%%%%%%%%%%%%%%%%%%%%%%%%%%%%%%%%%%%%%%%%%%%%%%%%%%%%%%%%%%%%%%%%%%%
\section*{Acknowledgements}

acknowledgements go here

%%%%%%%%%%%%%%%%%%%%%%%%%%%%%%%%%%%%%%%%%%%%%%%%%%%%%%%%%%%%%%%%%%%
\tableofcontents
\listoffigures
%%%%%%%%%%%%%%%%%%%%%%%%%%%%%%%%%%%%%%%%%%%%%%%%%%%%%%%%%%%%%%%%%%%

%%%%%%%%%%%%%%%%%%%%%%%%%%%%%%%%%%%%%%%%%%%%%%%%%%%%%%%%%%%%%%%%%%%
\chapter{Introduction}\label{intro}

\section{Overview}
\subsection{Problem statement}
This project proposes to analyse user’s searching behaviour in a context deprived of any cues 
that could allow them to exploit any previous experience in the task in order to achieve an 
optimal information foraging behaviour. 
User’s performance will be recorded and evaluated to see if it matches data in experiments in 
which this context is present. Findings could help provide insight on people’s choices when 
faced with a task that requires the same kind of skills that are required in information foraging 
with none of the context that they are presented with in experiments based on the same 
theories.
Finally this project aims at making user’s experience as enjoyable as possible for two reasons.
Firstly it will avoid users feeling like they are performing work, removing them further from a 
standard information foraging task. Secondly, in order to enhance the number of users that 
play the game as well as the amount of games played.

\subsection{The metaphore}


%%%%%%%%%%%%%%%%%%%%%%%%%%%%%%%%%%%%%%%%%%%%%%%%%%%%%%%%%%%%%%%%%%%
\chapter{Survey}\label{survey}

\section{Background theories }

\subsection{Optimal Foraging Theory}
In his chapter on the “Optimal Foraging Theory” Barry Sinevro (2006) provides us with a 
wealthy amount of field studies on animals in the wild, whose foraging efforts seem to be 
aimed towards maximising their energy gain per unit of time. This striving towards the 
implementation of optimal foraging techniques, in fact, represents a large part of the lives of 
many species and, in turn, a fundamental component in determining their adaptive success.
For instance, Sinevro tells us, the common shrew seems to always keep itself a mere few hours 
from death at all times. Because its small body size, the shrew cannot afford “the luxury of a 
thick layer of fat.” (Sinevro 2006) and has to constantly forage to keep itself alive. 
One of the most interesting studies proposed by Sinevro which can help us bring out some 
interesting points on searching behaviours, is the one on crows foraging on clams. In this 
study, Sinevro tells us about the foraging behaviour of the common crow (Corvus caurinus) in 
the intertidal. This kind of crow, has developed a technique to open clams which requires it to 
take a short flight and drop clams on some rocks below it to crack them open. The cost (in 
energy) of searching for clams, and the one of handling them is almost the same, however, the 
cost (in time) of performing the same activities is 4 times more expensive with regards to 
searching. It is then difficult to understand why crows would reject a large amounts of the 
clams they find, especially given the fact that searching seems to take up so much time. The 
reason for this behaviour is to be found in the “average net profitability of the clams as a 
function of size” (Sinevro 2006), which can be represented by the equation: 
\subsection{Information Foraging Theory}

\section{Previous studies}
\subsection{Searching behaviour}
\subsection{Gamification}

%%%%%%%%%%%%%%%%%%%%%%%%%%%%%%%%%%%%%%%%%%%%%%%%%%%%%%%%%%%%%%%%%%%
\chapter{Design}\label{design}

The design of Gold Digger went through different iterations through the course of its development (see 4.2 Heuristic Evaluation), however some design choices were made in order to keep the user experience consistent throughout the website and enhance clarity and ease of use.

\textbf{Separation between "game pages" and "site pages"}

Because Gold Digger is both a website and a game, it seemed appropriate to somehow separate the "game environment" from the more "site-like" features and pages, while still maintaining a certain degree of coherence between them. "Game pages" are the ones immediately relevant to the game in itself, the ones that the user will most likely enter while playing Gold Digger. These are: \textbf{the Game/Mine page, the General Shop page, the Home Page} and \textbf{the World Map Page}. The other pages are said to be "site pages" and they include pages like the About page and the Leaderboards, which a user is not very likely to access while in the middle of a play through. Furthermore "game pages" all require the user to be logged in to be accessed while the "site pages" do not. 
\\*
Notwithstanding this distinction, the user experience is not disjointed thanks to prominent recurring elements such as the nav bar on the top of the page and the landscape headline showing the name of each page.

\section{Site Map}

Navigation through the website is aided by a navbar located at the top of every page (with the exception of the 'game over' and 'end of the day'
pages). 
\begin{figure} [h] 
	\centering
           \makebox[\textwidth][c]{\includegraphics[width=1\textwidth]{nav.png}}
	\caption{Gold Digger nav bar}
           \label{navbar}
\end{figure}

From the navbar the user will be able to reach the following pages:

\begin{itemize}
  	\item The main page (by clicking on 'Gold Digger')
  	\item The World Map page (if signed in)
	\item The Leadeboards
	\item The About page
	\item The 'How to Play' / Tutorial page
	\item The 'Achievements' page
  	\item The User Profile page (if signed in)
\end{itemize} 

finally, the user will also be able to logout at any moment by using the 'Logout' button on the top right. However, if the user is currently in one of the mines,
clicking on one of these links will result on a warning message being displayed that will alert users that the gold gathered during the day will be lost if they leave
the mine before exhausting the time at their disposal and reaching the end of the day. 

\subsection{Home, Registreation and Login}


\begin{figure} [h] 
	\centering
           \makebox[\textwidth][c]{\includegraphics[width=1.2\textwidth]{homescreen.png}}
	\caption{Home Page}
           \label{homepage}
\end{figure}

On the landing page (homepage) users will be presented with a short \textbf {five-point explanation} of the game to quickly explain the mechancs of the game so that
users could start playing as soon as poosible. To so this, they will have to login through the form on the left or register by clicking on the 'Register' button. 
\begin{figwindow}%
[0, r, \includegraphics[width=.5\textwidth]{register.png},%
{ Registration modal}]
Clicking on the \textbf{ 'Register'} button will trigger a modal asking users to create a new username and password, as well as optionally entering their location and user
picture. Finally, the modal displays a disclaimer in order to both make sure that the users are aware of the limitations of accessing the website and its purpose. 
Users who wish to have more information about Gold Digger are redirected to the 'About' page or offered a link to directly write an email to the developer.
\\
Finally if a user enters wrong details or forgets to fill in a required field, (both for registration and login) an appropriate error message is clearly displayed for the user to see and try again. 
At present there is no limit to the number of attempts a user can make at logging in or registering.

\end{figwindow}

\subsection{World Map}

\begin{figure} [h] 
	\centering
           \makebox[\textwidth][c]{\includegraphics[width=1.2\textwidth]{worldmap.png}}
	\caption{World Map}
           \label{worldmap}
\end{figure}

Once users have registerd or have been logged in, they are automatically redirected to the world map so that they can start playing immediately. 
On the left of the page, users can see the euqipment and the amount of gold they have at the moment. The side item panel appears on the left column of the page, once the user has logged in. The panel shows the items that users have equipped, together with the total amount of gold in their possession. By hovering over each one of the items a tooltip will appear, showing the essential stats of each item. The presence of the side item panel has the function of both confirming the users that they are logged in and  reminding them that their game session has started  contributing to the uniformity of the user experience.
\begin{figwindow}%
[0, l, \includegraphics[width=.4\textwidth]{cali.png},%
{ California modal}]
 On the right hand side of the page users can access any of six locations: \textbf{California, Yukon, Brazil, South Africa, Scotland} and \textbf{Victoria}. 
Clicking on any of the locations will trigger a modal that displaying the scenery of the particular mine, some information about the gold digging history in that region (with links to Wikipedia articles), the cost of the mine and the amount of gold the user can expect to find. Starting from California and ending with Victoria, each mine is more expensive than the previous one 
but it also have a potentially higher gold yield. It is up to the player to enter the right mine at the right time.
\end{figwindow}

\subsection{Game Screen}
\begin{figure}[h] 
	\centering
           \makebox[\textwidth][c]{\includegraphics[width=1.2\textwidth]{mine.png}}
	\caption{Mine}
           \label{mine}
\end{figure}

Once the users enter a mine, the appropriate amount of gold is removed from the total and they can start digging. Each mine presents a similar structure, except for the landscape and the
amount of gold that can be found in each layer. On the top of the page we can see the landscape picture and the name of the location, as well as the number of mines that the user has dug into during the present day. As previously noted, on the left, we can find the equipment panel, in the middle we have the mine shaft and on the right have the \textbf {'Dig'} and \textbf{'Move'} as well as the yield of each layer that has been already dug. The left side panel contains here some more information than it does in the other pages where it appears:
\begin{itemize}
  	\item \textbf{Current gold}: the amount of gold gathered during the present day.
  	\item \textbf{Time remaining}: the amount of time (in units of time) remaining before the end of the day.
\end{itemize} 
Furthermore, because the left side panel's position is fixed, it will follow users as they get deeper and deeper in the mine, allowing them to always keep an eye on the game state without having to go back to the top of the page to check how many units of time they have left before the end of the day. Because the 'Current gold' is summed to the 'Total Gold' only at the end of the day, the total amount of gold, together with the number of days users have been digging without encountering a game over, is displayed in a small box on the top right of the page.
 The options available to the user and an example of gameplay are detailed in section 3.2.

\subsection{General Store}

\begin{figure} [h] 
	\centering
           \makebox[\textwidth][c]{\includegraphics[width=1.2\textwidth]{generalstore.png}}
	\caption{General Store}
           \label{fig: generalstore}
\end{figure}

At the end of each day (by clicking on \textbf{'shop'}), or from the world map screen (by clicking on \textbf{'Buy stuff'}), users can access the \textbf{General Store} (see fig. \ref{fig: generalstore}). From here they are able to browse and purchase new items to help them in their digging. There are three groups of objects that the user can choose from \textbf{Scanning Equipment, Digging Equipment} and \textbf{Vehicles}. Each of these groups contains five different items with different cost and stats. Users are able to see the image associated with each item as well as its cost,  stats and a short description in the small tex box underneath. If users decide to upgrade one of their item groups (Scanning, Digging or Vehicle) they can do so by clicking on the \textbf{'Upgrade'} button. At this point they will be presented with a small modal that reviews the item's stats, together with its picture, asking the user to confirm the purchase. If the users don't have enough gold to make the purchase, or if the purchase would cause them to not have enough money to enter the cheapest mine, an appropriate alert message is displayed, explaining why the purchase is not possible.

The choice to allow users to \emph{upgrade} rather that \emph{buy} items has been made because each of the items is better than the previous one in every respect and thus there is no reason why a user would chose to equip an item that they previously purchased, since this would not bring any advantage at all. For instance, with digging tools, their cost, dig cost (the amount the user has to pay in time units to dig once) and extraction power (the percentage of gold that the user will be able to extract given a certain yield), are all increased from the least, to the most expensive. However, it would be easy to add items that have different combinations of these parameters and add, for example, a digging tool that, at a higher cost per dig also returns a higher percentage of gold.
Finally, upon purchase, the new item is immediately added to the left side panel and the appropriate amount of money removed from the user's total through an AJAX call.

 \subsection{Leaderboards}

\begin{figure} [h] 
	\centering
           \makebox[\textwidth][c]{\includegraphics[width=1.2\textwidth]{leaderboards.png}}
	\caption{Leaderboards}
           \label{fig: leaderboards}
\end{figure}

On the \textbf{Leaderboards} (see fig. \ref{fig: leaderboards}) page, users are able to check their performance and compare it to the other players'. There are four parameters by which users can be ranked, plus an achievements board that has no specific ranking criteria: 

\begin{itemize}
  	\item \textbf{Average}: the total amount of gold ever dug by the player divided by the total amount of mines that were dug into.
  	\item \textbf{Max Gold}: the maximum amount of Gold ever dug (the maximum amount of gold ever reached)
	\item \textbf{Days Worked}: the total amount of days of digging completed by the player (not zeroed on game over)
	\item \textbf{All Time Gold}: the cumulative total amount of gold dug by the player  (not zeroed on game over)
	\item \textbf{Achievements}: shows the achievements gained by each player in no particular order (achievements are not ranked)
\end{itemize} 

Each row of the the tables (excluding the 'Achievement' table) diplays the following parameters: Rank, Thumbnail, Username, Equipment (showing the image for each of the three item groups), Average Gold per Mine, All Time Gold Dug, Max Gold Reached, Days Dug.

\subsection{Tutorial}

\begin{figure} [h] 
	\centering
           \makebox[\textwidth][c]{\includegraphics[width=1.2\textwidth]{tutorial.png}}
	\caption{Tutorial}
           \label{fig: tutorial}
\end{figure}

Because it is important for users to have a clear idea of the way the game works, all effort has been made to employ a visual approach that would be quick and easy to understand. To this end, upon entering, the \textbf{'How to Play'} page, a tour of the main game features is automatically launched. The javascript library \textbf{Trip.js} (see 4.4.6) highlights and points at the different parts of the game screen in order to get a quick visual explanation of the main game mechanics. Each of the labels has 'Next', and 'Previous' link, in order to let users go back to a previous point or skip explanations they already viewed.

\subsection{Achievements}

\begin{figure} [!hb] 
	\centering
           \makebox[\textwidth][c]{\includegraphics[width=1.2\textwidth]{achievements.png}}
	\caption{Achievements display}
           \label{fig: achievements}
\end{figure}

The \textbf{'Achievements'} page dispays all the achievemts badges together with their name, the condition that triggers them and their image. This page is needed in order for the users to be able to check which achievements they can aim for while playing. Two additional achievements (fig. \ref{fig: specialachievements}) are not displayed here and are given only under special conditions:

\begin{figure}[!h]
        \centering
        \begin{subfigure} [!h] {0.4\textwidth}
                \centering
                \includegraphics [width=0.4\textwidth] {awesome.png}
                \caption{You helped testing!}
                \label{}
        \end{subfigure}
        \space
        \space
        \begin{subfigure} [!h] {0.4\textwidth}
                \centering
                \includegraphics [width=0.4\textwidth] {banana.png}
                \caption{You found the Easter Egg!}
                \label{}
        \end{subfigure}
        \caption{Special Achievements}
        \label{fig: specialachievements}
\end{figure}

\begin{itemize}
  	\item \textbf{Awesomeness}: given to the people who tested the "beta" version of the site.
  	\item \textbf{Banana}: given to people who managed to find the game's Easter Egg
\end{itemize} 

Achievements ans special achievements have been incorporated in order to both add an extra game feature to Gold Digger which wouldn't change the game mechanics too much an to give players an incentive to keep playing, especially since there is no particular "winning condition", fulfilled which a player has completed the game.  

\subsection{About Page}

In this page, users are able to find  the developer and supervisor's contact details as well as reading about the reasons behind the creation of Gold Digger and acknowledgements of the authors of some of the material used in the site.

$4+4 \in $ \cite{BK08}

\section{Walkthrough}

\begin{figure}[!h]
        \centering
        \begin{subfigure} [h] {0.6\textwidth}
                \centering
                \includegraphics [width=1\textwidth] {worldmap.png}
        \end{subfigure}
        \space
        \space
        \begin{subfigure} [h] {0.3\textwidth}
                \centering
                \includegraphics [width=1\textwidth] {cali.png}
        \end{subfigure}
\end{figure}

(1) Jill, the user, has just logged in (or registered) and has been redirected to the World Map. Her starting Equipment and Gold are as follows:
\begin{itemize}
	\item \textbf{Gold}: 100
  	\item \textbf{Scanning Tool}:  Oil Lamp (accuracy: 20\%, visibility: 2)
  	\item \textbf{Digging Tool}: Spoon (dig cost: 5, gold extracted 30\%)
	\item \textbf{Vehicle}: Boots (move cost: 10)
\end{itemize} 

On the map, Jill clicks on California and a modal appears, from which she can see see that, although the mine doesn't yield much gold, it is relatively cheap to access it (she only needs to pay 40 gold nuggets to enter it).

(2) After clicking on 'Go to Mine', 40 gold nuggets are removed from Jill's total amount of gold and she is presented with a mine where no layer has been dug. On the top left hand side Jill can see the name of the location she is in and the number of the mine (now 1) whereas on the right hand side she can check her total amount of gold (now 60) and the number of days she has been digging. Thanks to her Oil Lamp she can see specks of gold in the first two layers but, because the lamp only has a 20\% accuracy, it is possible that the amount of flecks that can be seen is deceiving. 
\begin{itemize}
	\item \textbf{Current gold}: 0
  	\item \textbf{Time remainingl}: 100
  	\item \textbf{Mine}: 1
	\item \textbf{Day}: 1
\end{itemize} 

(3) Jill decides to dig through the first couple of layers layer of the mine by clicking on the 'Move' button on the right hand side of the mine shaft. She gains 17 gold nuggets and consumes 10 units of time (since each digging operation costs 5 units of time)
\begin{itemize}
	\item \textbf{Current gold}: 17
  	\item \textbf{Time remainingl}: 90
  	\item \textbf{Mine}: 1
	\item \textbf{Day}: 1
\end{itemize} 

(4) At this point Jill is not sure she might get much more gold from this mine and she doesn't want to spend precious time digging through layers of the mine that she has no information about. For this reason she clicks on 'Move' to get to a new mine (in the same location). To do this she must spend 10 units of time
\begin{itemize}
	\item \textbf{Current gold}: 17
  	\item \textbf{Time remainingl}: 80
  	\item \textbf{Mine}: 2
	\item \textbf{Day}: 1
\end{itemize} 

\begin{figure}[!h]
        \centering
        \begin{subfigure} [h] {0.6\textwidth}
                \centering
                \includegraphics [width=1\textwidth] {endofday.png}
        \end{subfigure}
        \space
        \space
        \begin{subfigure} [h] {0.2\textwidth}
                \centering
                \includegraphics [width=1\textwidth] {bronze.png}
        \end{subfigure}
\end{figure}

(5) Jill continues to dig through some more mines and after a while she runs out of units of time. At this point, she is presented with the 'End of the Day' screen that sums up her progress during the day as follows:
\begin{itemize}
	\item \textbf{Today's gold}: 94 
  	\item \textbf{Total gold}: 94
  	\item \textbf{Mines dug}: 6 
\end{itemize} 
Jill also receives the 'Bronze Coin' achievement for having dug more than 50 gold nuggets. From here Jill decides to go to the shop and see if she can purchase some new items

\begin{figure}[!h]
        \centering
        \begin{subfigure} [h] {0.6\textwidth}
                \centering
                \includegraphics [width=1\textwidth] {generalstore.png}
        \end{subfigure}
        \space
        \space
        \begin{subfigure} [h] {0.3\textwidth}
                \centering
                \includegraphics [width=1\textwidth] {nogold.png}
        \end{subfigure}
\end{figure}

(6) Once in the shop Jill tries to purchase better scanning equipment, so she clicks on the 'Upgrade' button next to the Oil Lamp. Unfortunately she doesn't have enough gold to make this purchase yet, so a message alerts her that she would need to do some more mining if she wants to be able to purchase the item.

\begin{figure}[!h]
        \centering
        \begin{subfigure} [h] {0.6\textwidth}
                \centering
                \includegraphics [width=1\textwidth] {yukon.png}
        \end{subfigure}
        \space
        \space
        \begin{subfigure} [h] {0.3\textwidth}
                \centering
                \includegraphics [width=1\textwidth] {yuki.png}
        \end{subfigure}
\end{figure}

(7) To gather more gold, Jill goes back to the California mines and, as soon as she has 100 gold nuggets, she takes a gamble and spends them on accessing the Yukon mine. However, she doesn't perform very well and, at the end of the day she owns less than 40 gold nuggets.
\begin{figure} [h] 
	\centering
           \makebox[\textwidth][c]{\includegraphics[width=1\textwidth]{gameover.png}}
\end{figure}
Because she doesn't have enough money to enter any of the mine, Jill loses her first game. However, clicking on 'Back to Map' will restore her to the initial conditions, if she bought any items they wold be lost.
\section{System Architechture}

\begin{figure} [h] 
	\centering
           \makebox[\textwidth][c]{\includegraphics[width=1\textwidth]{architecture.png}}
\end{figure}

Gold Digger is based on a 3-tier architecture. Each user is able to create a player account (client) containing all of the details entered upon registration, together with a set of game variables, set to their starting values. The client side renders HTML5 and CSS3 elements and graphics while updates to the game states are done mostly through AJAX, in order to avoid reloading the page. The game logic is coded in the middleware using the python-based Django framework (see \ref {subsec:Django}). All of the user profiles as well as the other game objects and achievements are stored in the database while the corresponding graphics, as well as the rest of the game graphics are stored in the project's static and media folders. Finally, the middleware continuously updates a log file in order to monitor user behaviour and gather data to be parsed and analysed at a later stage.
\section{ER Model}

\begin{figure} [h] 
	\centering
           \makebox[\textwidth][c]{\includegraphics[width=0.7\textwidth]{er.png}}
	\caption{The Gold Digger ER diagram}
\end{figure}

The above diagram describes the entity-relationship model for Gold Digger. There are 5 entities in Gold Digger:
\begin{itemize}
	\item \textbf{User Profiles}: this entity contains several of the users' game stats, like the amount of gold at their disposal and the average amount of gold dug per mine as well as a reference to the Django object 'user' containing all the information regarding the user's password, username and email. This is necessary in order to take advantage of Django's very efficient user management system and keep the website as safe as possible 
  	\item \textbf{Digging Equipment}: this entity contains each of the tools' name, image, price and description. However, the most important attributes of this entity are the \textit{modifier} and the \textit{time modifier}. The\textit{modifier} attribute is used by the game mechanics to calculate how much gold a user is able to extract from a given layer of the mine. It is a \textit{float} number between 0.2 and 0.8  that will be multiplied by each of the amounts of gold in each layer. This way if the user is using, for instance, a shovel, she will be able to extract only 30\% of the gold that is actually present in that layer (see \ref{subsec:code}). The \textit{time modifier} attribute, on the other hand stores the digging cost per digging operation of each of the tools and it is an integer that goes from 5 to 1 in decreasing order from the worst tool to the best.
  	\item \textbf{Scanning Equipment}: this entity is quite similar to Digging Equipment but it does not include the \textit{time modifier} attribute. The \textit{modifier} attribute of this entity is responsible for calculating the cue array, that is in turn responsible for determining the amount of gold flecks that will be shown in each of the mine's layers. Furthermore, because this attribute holds only float values from 0.2 to 0.8, a simple multiplication by 10 is used  to determine the number of layers in which those specs are actually visible  (see \ref{subsec:code}).
	\item \textbf{Vehicles}: 1
	\item \textbf{Achievements}: 1
\end{itemize} 

\section{Graphics}


%%%%%%%%%%%%%%%%%%%%%%%%%%%%%%%%%%%%%%%%%%%%%%%%%%%%%%%%%%%%%%%%%%%
\chapter{Impementation}\label{implementation}
\section{Development Methods}
\section{Heuristic Evaluation}
\section{Testing}
\subsection{Unit Testing}
\subsection{Live Testing}
\section{Technologies}
\subsection{Django}
\label{subsec:Django}
\subsection{Python}
\subsection{HTML5}
\subsection{CSS3}
\paragraph{Animate.css}
\subsection{Twitter Bootstrap}
\subsection{Javascript and JQuery}
\paragraph{Trip.js}
\paragraph{Animatenumbers.js}
\subsection{AJAX}
\subsection{Code}
\label{subsec:code}

%%%%%%%%%%%%%%%%%%%%%%%%%%%%%%%%%%%%%%%%%%%%%%%%%%%%%%%%%%%%%%%%%%%
\chapter{Results and Data}\label{results}
\section{Data Logged}
\section{Data Analysis}
\section{Results}
\section{Reflection}







%%%%%%%%%%%%%%%%%%%%%%%%%%%%%%%%%%%%%%%%%%%%%%%%%%%%%%%%%%%%%%%%%%%

\appendix % first appendix
%%%%%%%%%%%%%%%%%%%%%%%%%%%%%%%%%%%%%%%%%%%%%%%%%%%%%%%%%%%%%%%%%%%
\chapter{First appendix}

\section{Section of first appendix}

%%%%%%%%%%%%%%%%%%%%%%%%%%%%%%%%%%%%%%%%%%%%%%%%%%%%%%%%%%%%%%%%%%%
\chapter{Second appendix}

%%%%%%%%%%%%%%%%%%%%%%%%%%%%%%%%%%%%%%%%%%%%%%%%%%%%%%%%%%%%%%%%%%%
% it is fine to change the bibliography style if you want
\bibliographystyle{plain}
\bibliography{mproj}
\end{document}
